\documentclass[11pt,a4paper,oneside]{report}
\usepackage[utf8]{inputenc}
\usepackage[ngerman]{babel}
\usepackage{hyperref}
\usepackage{fancyvrb}
\usepackage{listingsutf8}
\usepackage{xcolor}
\usepackage{DejaVuSerif}
\usepackage[T1]{fontenc}

\hypersetup{
  colorlinks,
  linkcolor=blue,
  urlcolor=blue
}

\lstset{
  language=Java,
  basicstyle=\ttfamily\small,
  numberstyle=\footnotesize,
  numbers=left,
  showspaces=false,
  showtabs=false,
  fancyvrb=true,
  backgroundcolor=\color{gray!8},
  keywordstyle=\color{blue!80!black!100},
  identifierstyle=,
  commentstyle=\color{green!50!black!100},
  stringstyle=\ttfamily\color{orange},
  morekeywords={enum},
  breaklines=true,
  breakatwhitespace=true,
  frame=single,
  captionpos=b,
  tabsize=2,
  rulecolor=\color{black!30},
  title=\lstname,
  framextopmargin=2pt,
  framexbottommargin=2pt,
  literate=%
    {Ö}{{\"O}}1
    {Ä}{{\"A}}1
    {Ü}{{\"U}}1
    {ß}{{\ss}}2
    {ü}{{\"u}}1
    {ä}{{\"a}}1
    {ö}{{\"o}}1
}

\begin{document}

\title{OOPS-"Ubersetzer}
\date{SoSe 2012}
\author{Dirk Evers, Paul Koch}

\maketitle

\tableofcontents

\chapter{TRUE und FALSE}
\section{Einführung}
In diesem Kapitel führen wir die Begriffe \verb|TRUE| und \verb|FALSE| als erste boolsche Werte ein.\\

\section{Lexikalische Analyse}
Damit die Symbole  \verb|TRUE| und \verb|FALSE|  eingelesen werden können müssen sie zuerst in  \verb|Symbol.java| definiert werden.
\lstinputlisting[firstnumber=21, linerange={21-21}]{../oopsc/src/Symbol.java}
Nun da wir Symbole haben mit denen OOPSC arbeiten kann, definieren wir in \verb|LexicalAnalysis.java|, dass er die Symbole anlegen soll, wenn er \verb|TRUE| bzw. \verb|FALSE| ließt.
\lstinputlisting[firstnumber=89, linerange={89-92}]{../oopsc/src/LexicalAnalysis.java}

\section{Syntaktische Analyse}
In der Syntax von OOPS gehören  \verb|TRUE| und \verb|FALSE| zu den Literalen, also müssen wir in \verb|SyntaxAnalysis.java| Sie als LiteralExpression in \verb|literal()| einfügen. Da nun alle wieteren Komponenten wie BoolType bereits bekannt sind, könne wir sie verwenden und  \verb|TRUE| und \verb|FALSE| sind erfolgreich eingebunden.
\lstinputlisting[firstnumber=646, linerange={646-655}]{../oopsc/src/SyntaxAnalysis.java}


\chapter{ELSE und ELSEIF}
\section{Einführung}
In diesem Kapitel wird das If-Statement um einen möglichen else-Pfad und mehreren möglichen elseIf-Pfaden erweitert.
Dazu werden die Bezeichner dem Lexika hinzugefügt und die Syntax und die Kontext Analyse um die else- bzw. elseIf-Pfade erweitert. Ebenso wird nun auch der generierte Code erweitert.

\section{Lexikalische Analyse}
Zuerst werden die Symbole in  \verb|Symbol.java| definiert.\\
Dafür wird folgendes der enum Id in \verb|Symbo.java| hinzugefügt:
\lstinputlisting[firstnumber=22, linerange={22-22}]{../oopsc/src/Symbol.java}
In der lexikalischen Analyse in \verb|LexicalAnalysis.java| wird ergänzt, dass beim einlesen von \verb|ELSE| das Symbol ELSE und bei 
\verb|ELSEIF| das Symbol ELSEIF erzeugt wird.\\
Dazu wird LexicalAnalysis in \verb|LexicalAnalysis.java| erweitert:
\lstinputlisting[firstnumber=93, linerange={93-96}]{../oopsc/src/LexicalAnalysis.java}

\section{Syntaktische Analyse}
Bevor mit der syntaktischen Analyse begonnen werden kann muss  \verb|IfStatement.java|
durch elseStatements erweitert werden. Statements für die ELSEIF-Pfade sind nicht nötig, genauer wird darauf noch später eingegangen.
\lstinputlisting[firstnumber=13, linerange={13-16}]{../oopsc/src/IfStatement.java}
Für spätere Tests ist es auch sinnvoll die print() Funktion zu erweitern.
\lstinputlisting[firstnumber=90, linerange={90-99}]{../oopsc/src/IfStatement.java}
\vspace{3mm}
Die Syntax einer If-Anweisung sieht wie folgt aus:\\ \verb|IF relation THEN statements { ELSEIF relation THEN statements }|\\
\verb|[ ELSE statements ] END IF|\\  bisher ist vorhanden: \verb| IF relation THEN statements END IF|.\\
Als erstes müssen die IF-Statements zus"atzlich bei ELSE, bzw ELSEIF abgebrochen werden.
Entweder man ersetzt dafür die Statements hier durch einen neuen Statement-Typ ersetzt oder Statements wird erweitert.
Da  ELSE und ELSEIF in keinen anderen Kontext Verwendung findet, wird hier einfach Statements erweitert.
\lstinputlisting[firstnumber=361, linerange={361-368}]{../oopsc/src/SyntaxAnalysis.java}
Die restlichen Erg"anzungen der SyntaxAnalyse finden in \verb|statement()| unter dem Fall \verb|IF| statt.
Nach dem hinzufügen der IF-Statements muss nun überprüft werden, was als nächstes Zeichen folgt, ist es ELSE, bzw. ELSEIF, muss es verbraucht werden und weitere Statements für die ELSE-Statements, bzw ELSEIF-Statements hinzugefügt werden.
Dabei ist zu beachten, das nur ELSEIF folgende ELSEIF und/oder ELSE zulässt.
Da ELSEIF ebenso wie IF im ELSE-Pfad funktioniert, kann es auch so verwendet werden.
Dazu ver"andert man Das Symbol ELSEIF zu IF, erzeugt daraus ein \verb|ifStatement()| und fügt dieses zu den elseStatements hinzu.\\  Hierfür wird in statement der IF-Fall in \verb|SyntaxAnalysis.java| wie folgt erweitert:
\lstinputlisting[firstnumber=395, linerange={395-399}]{../oopsc/src/SyntaxAnalysis.java}
Dadurch werden Rekursiv weitere ELSEIF-Pfade möglich.	
Falls jedoch kein ELSEIF noch vorhanden ist, muss noch der ELSE-Fall geprüft werden.
Wenn das Symbol ELSE erkannt wird, werden die folgenden Statements den ELSE-Statements hinzugefügt.
\lstinputlisting[firstnumber=400, linerange={400-404}]{../oopsc/src/SyntaxAnalysis.java}

\section{Kontext Analyse}
Die Kontext Analyse findet in der Regel in der dazugehörigen Java-Datei statt.
In \verb|IfStatement.java| ist bereits eine Analyse für das Statement mit IF-Pfad vorhanden. Diesen müssen noch mit dem ELSE-Pfad ergänzen. Da IFELSE ersetzen wurde ist die Einbettung vollständig und muss nicht weiter Beachtet werden.
\lstinputlisting[firstnumber=39, linerange={39-43}]{../oopsc/src/IfStatement.java}

\section{Ausführender Code}
Auch hier wird auf den bereits bestehenden Code aufgesetzt.
Zuerst wird ein weiteres Label für den ELSE-Pfad benötigt.
\lstinputlisting[firstnumber=109, linerange={109-109}]{../oopsc/src/ifStatement.java}
Der Sprung falls die Bedingung nicht zutrifft muss auf den ELSE-Pfad gehen.
\lstinputlisting[firstnumber=116, linerange={116-116}]{../oopsc/src/ifStatement.java}
Um einen Code-Block zu erzeugen, der im normalen Fluss nicht erfasst wird, wird nach den IF-Statements ein Sprung zum "If END"-Label hinzugefügt. Damit dieser Block nun als ELSE-Behandlung erkannt wird, kommt das ELSE-Label hinzu.
Nun folgen die ELSE-Statements.
\lstinputlisting[firstnumber=121, linerange={121-128}]{../oopsc/src/ifStatement.java}

\chapter{Boolesche Arithmetik}
\section{Einführung}
In diesem Kapitel wird die Boolsche Arithmetik um AND, OR und NOT erweitert.
Dazu werden wieder die Bezeichner der Lexika hinzugefügt ,die Syntax und die Kontext Analyse erweitert und der zu generierende Code definiert.  Diesesmal sind die Syntax und der Code jedoch eigenständig. 

\section{Lexikalische Analyse}
Zuerst werden die Symbole in  \verb|Symbol.java| definiert.\\
Dafür wird folgendes der enum Id in \verb|Symbo.java| hinzugefügt:
\lstinputlisting[firstnumber=23, linerange={23-23}]{../oopsc/src/Symbol.java}
In der lexikalischen Analyse in \verb|LexicalAnalysis.java| wird ergänzt, wie die Token erstellt werden.\\
Dazu wird LexicalAnalysis in \verb|LexicalAnalysis.java| erweitert:
\lstinputlisting[firstnumber=97, linerange={97-100}]{../oopsc/src/LexicalAnalysis.java}

\section{Syntaktische Analyse}
\subsection{NOT}
Die Boolsche Ausdruck NOT ist mit dem Ausdruck Minus zuvergleichen, beide negieren den Literalen-Ausdruck im Anschluss.
Der Unterschied wird erst in der Kontext Analyse.
\lstinputlisting[firstnumber=597, linerange={597-608}]{../oopsc/src/SyntaxAnalysis.java}

\subsection{OR}
Die Syntax von OR ist der von der Multiplikation oder auch Addition ähnlich. 
Nur dass die nächst kleineren Ausdrücke Relationen sind, da OR die nächst höhere  Bindung hat.
Da AND die nächst höhere Bindung hat, wird der aufrug dort sein,siehe AND.
\lstinputlisting[firstnumber=513, linerange={513-528}]{../oopsc/src/SyntaxAnalysis.java}
\subsection{AND}
Die Syntax von AND ist ebenfalls den von der Multiplikation oder auch Addition ähnlich. 
Nur dass die nächst kleineren Ausdrücke die eben definierten logicalOR sind.
\lstinputlisting[firstnumber=597, linerange={597-608}]{../oopsc/src/SyntaxAnalysis.java}
Nun fehlen noch die Aufrufe von logicalAND. Diese sind überall dort, wo vorher die Relationen erwartet wurden. mit ausnahme des eben definierten logicalOR.
IF-Statements:\\
\lstinputlisting[firstnumber=391, linerange={391-391}]{../oopsc/src/LexicalAnalysis.java}
WHILE-Statements:\\
\lstinputlisting[firstnumber=410, linerange={410-410}]{../oopsc/src/LexicalAnalysis.java}
und zu letzt im Literal:
\lstinputlisting[firstnumber=678, linerange={678-678}]{../oopsc/src/LexicalAnalysis.java}

\section{Kontext Analyse}
\subsection{NOT}
Für NOT wird UnaryExpression erweitert.
Wenn als Operator NOT erkannt wird, soll ein boolType erwartet werden.
\lstinputlisting[firstnumber=34, linerange={34-38}]{../oopsc/src/UnaryExpression.java}
\subsection{AND und OR}
Beide Ausdrücke sind BinaryExpression und bei Beiden müssen die Sieten vom Typ boolType sein.
\lstinputlisting[firstnumber=52, linerange={52-59}]{../oopsc/src/BinaryExpression.java}



\section{Ausführender Code}
\subsection{NOT}
Um den Ausdruck zu verneinen wird ISZ auf das Ergebniss verwendet.
ISZ prüft auf 0 und gibt dann 1 zurück, somit wird der Wert negiert.
\lstinputlisting[firstnumber=114, linerange={114-119}]{../oopsc/src/UnaryExpression.java}
\subsection{AND und OR}
Für AND und OR existieren bereits Befehle, welche hier nur aufgerufen werden müssen.
\lstinputlisting[firstnumber=385, linerange={385-394}]{../oopsc/src/BinaryExpression.java}

\chapter{Klasse Boolean}
\section{Einführung}
In diesem Kapitel fügen wir den Typ Boolean hinzu.
Heirfür muss zunächst das Objekt erstellt, die Grösse festgelegt und  den Deklarationen hinzufügen werden.
Ausserdem muss \verb|box< , bzw \verb|unBox|, angepasst werden.

\section{Implementierung}
Zuerst muss die Deklaration boolClass in \verb|ClassDeclaration.java| erstellt werden.
\lstinputlisting[firstnumber=37, linerange={37-40}]{../oopsc/src/ClassDeclaration.java}
Wie auch beim Integer muss die Objektgröße in der contextAnalyse definiert werden.
\lstinputlisting[firstnumber=37, linerange={54-57}]{../oopsc/src/Program.java}
 Außerdem muss boolClass den Deklarationen hinzugefügt werden.
\lstinputlisting[firstnumber=69, linerange={69-70}]{../oopsc/src/Program.java}
\section{Boxen}
Jetzt können bereits Boolean-Objekte angelegt werden, aber nicht geboxt, bzw. unboxt werden.
Dazu müssen noch \verb|BoxExpression.java| und \verb|UnBoxExpression.java| erweitert werden. Dabei sollte auch boolClass als Operand zugelassen werden:
\lstinputlisting[firstnumber=17, linerange={17-29}]{../oopsc/src/UnBoxExpression.java}

\lstinputlisting[firstnumber=23, linerange={23-38}]{../oopsc/src/BoxExpression.java}

\chapter{ Mehrere eigene Klassen}
\section{Einführung}
In diesem Kapitel wird Das Program um weitere Klassen erweitert.
Dazu muss die Syntax und der Code von Program erweitert werden.
Auch können nun Boolean und Integer diesen Klassen hinzugefügt werden, statt gesondert behandelt zu werden.

\section{Klassen Deklaration}
Damit alle Deklarationen zur Kontext-Analyse bekannt sind, muss das Deklarieren und die Analyse in ClassDeclaration getrennt werden,
 sodass  eine neue Methode \verb|contextAnalysisForBody(Declaration declarations)| verfügbar ist.
\lstinputlisting[firstnumber=218, linerange={218-231}]{../oopsc/src/ClassDeclaration.java}

\section{mehrere Klassen im Programm}
In Programm muss zuerst die classDeclaration durch eine Liste von classDeclaration's ersetzt werden.
\lstinputlisting[firstnumber=9, linerange={9-12}]{../oopsc/src/Program.java}
Ebenso muss der Konstruktor angepasst werden. In diesem Code werden nun zwei Konstruktoren geführt. Einer für nur eine Klasse und einer führ eine Liste von Klassen.
\lstinputlisting[firstnumber=22, linerange={22-41}]{../oopsc/src/Program.java}
Da nun das Programm bereits mehrere Klassen führen kann, muss nun angepasst werden, was damit getan werden muss.
Dazu ist die Bearbeitung zur vorherigen Klasse ähnlich. Nur dass nun die Deklaration und Kontext-Analyse getrennt ausgeführt werden müssen, damit sämtliche Definitionen der Klassen und Methoden der Klassen wärend der ganzen Kontext-Analyse vorhanden ist.
\lstinputlisting[firstnumber=98, linerange={98-122}]{../oopsc/src/Program.java}

Da Boolean und Integer ebenfalls Klassen sind können sie nun mit den anderen Behandelt werden, so entfällt der bisherige Code in \verb|Program.java| von Integer und Boolean stattdessen werden sie der Liste der Klassen hinzugefügt.
\lstinputlisting[firstnumber=69, linerange={69-70}]{../oopsc/src/Program.java}
Daraufhin müssen sie aber mit \verb|null| initialisiert werden.
\lstinputlisting[firstnumber=86, linerange={86-94}]{../oopsc/src/Program.java}

\section{Syntax}
Zum Schluss muss noch die Syntax angepasst werden, sodass auch mehrere Klassen für ein Programm gelesen werden können.
Dazu fügen wir der Funktion \verb|parse()| eine Schleife hinzu, welche mit Hilfe des Symbols \verb|Class| nach Klassen sucht.
Die Notwendigkeit und Eigenschaften von Main sind hier nicht nötig, da sie wie vorher in \verb|Programm.java| berücksichtigt werden.
\lstinputlisting[firstnumber=742, linerange={742-750}]{../oopsc/src/SyntaxAnalysis.java}

\section{Ausführender Code}
Der meiste Code ist bereits über die Deklarationen abgedeckt, somit sind die Aufrufe bereits abgeschlossen.
Da ebenfalls bereits für Klassen eine vollständige \verb|generateCode| Methode existiert, muss Diese nurnoch für alle Klassen ausgeführt werden.
\lstinputlisting[firstnumber=197, linerange={197-201}]{../oopsc/src/Program.java}


\chapter{Parameter f"ur Methoden}
\section{Einführung}
In diesem Kapitel wird Das Program um weitere Klassen erweitert.
Dazu muss die Syntax und der Code von Program erweitert werden.
Auch können nun Boolean und Integer diesen Klassen hinzugefügt werden, statt gesondert behandelt zu werden.

\section{Klassen Deklaration}
Damit alle Deklarationen zur Kontext-Analyse bekannt sind, muss das Deklarieren und die Analyse in ClassDeclaration getrennt werden,
 sodass  eine neue Methode \verb|contextAnalysisForBody(Declaration declarations)| verfügbar ist.
\lstinputlisting[firstnumber=218, linerange={218-231}]{../oopsc/src/ClassDeclaration.java}

\section{mehrere Klassen im Programm}
In Programm muss zuerst die classDeclaration durch eine Liste von classDeclaration's ersetzt werden.
\lstinputlisting[firstnumber=9, linerange={9-12}]{../oopsc/src/Program.java}
Ebenso muss der Konstruktor angepasst werden. In diesem Code werden nun zwei Konstruktoren geführt. Einer für nur eine Klasse und einer führ eine Liste von Klassen.
\lstinputlisting[firstnumber=22, linerange={22-41}]{../oopsc/src/Program.java}
Da nun das Programm bereits mehrere Klassen führen kann, muss nun angepasst werden, was damit getan werden muss.
Dazu ist die Bearbeitung zur vorherigen Klasse ähnlich. Nur dass nun die Deklaration und Kontext-Analyse getrennt ausgeführt werden müssen, damit sämtliche Definitionen der Klassen und Methoden der Klassen wärend der ganzen Kontext-Analyse vorhanden ist.
\lstinputlisting[firstnumber=98, linerange={98-122}]{../oopsc/src/Program.java}

Da Boolean und Integer ebenfalls Klassen sind können sie nun mit den anderen Behandelt werden, so entfällt der bisherige Code in \verb|Program.java| von Integer und Boolean stattdessen werden sie der Liste der Klassen hinzugefügt.
\lstinputlisting[firstnumber=69, linerange={69-70}]{../oopsc/src/Program.java}
Daraufhin müssen sie aber mit \verb|null| initialisiert werden.
\lstinputlisting[firstnumber=86, linerange={86-94}]{../oopsc/src/Program.java}

\section{Syntax}
Zum Schluss muss noch die Syntax angepasst werden, sodass auch mehrere Klassen für ein Programm gelesen werden können.
Dazu fügen wir der Funktion \verb|parse()| eine Schleife hinzu, welche mit Hilfe des Symbols \verb|Class| nach Klassen sucht.
Die Notwendigkeit und Eigenschaften von Main sind hier nicht nötig, da sie wie vorher in \verb|Programm.java| berücksichtigt werden.
\lstinputlisting[firstnumber=742, linerange={742-750}]{../oopsc/src/SyntaxAnalysis.java}

\section{Ausführender Code}
Der meiste Code ist bereits über die Deklarationen abgedeckt, somit sind die Aufrufe bereits abgeschlossen.
Da ebenfalls bereits für Klassen eine vollständige \verb|generateCode| Methode existiert, muss Diese nurnoch für alle Klassen ausgeführt werden.
\lstinputlisting[firstnumber=197, linerange={197-201}]{../oopsc/src/Program.java}


\chapter{Ergebnisse f"ur Methoden}
\include{ch7_return}

\chapter{Ausnahmebehandlung}
\include{ch8_try}

\chapter{Vererbung}
\include{ch9_vererbung}

\chapter{Speicherbereinigung}
\section{Einf"uhrung}
Die ganze Zeit lang wurde der \verb|Push only Stack| verwendet.  Obwohl dieser
recht leicht zu implementieren ist, hat dieser den Nachteil, dass er irgendwann
voll wird.

In dieser Aufgabe wird ein \verb|Copy Collector| mit Stackverbrauch
implementiert. Dieser verwendet zwei Heaps. Es wird, wie beim \verb|Push only Stack|,
ein Heap sequentiell gef"ullt, bis dieser voll ist.

Wird beim erstellen eines Objekts detektiert, dass nicht genug Platz vorhanden
ist, so wird der \verb|Garbage Collector| aktiviert.
Dieser kopiert alle erreichbaren Objekte von aktuellen Heap auf den anderen und
arbeitet auf diesem weiter. Ist der Heap danach immer noch voll, so kann auch
der \verb|Garbage Collector| nicht mehr helfen und das Programm terminiert mit
einem Fehler.

Es stellt sich nun ein Problem, wie die erreichbaren Objekte ermittelt werden
k"onnen.  Dieses Problem l"asst sich l"osen, indem eine Wurzelmenge gef"uhrt
wird, die erstmal nur die Objekte enth"alt, die direkt vom Stack referenziert
werden. Dann werden diese Objekte auf den neuen Heap mit einer
\verb|clone|-Methode kopiert, die jede Klasse definiert.  Diese Methode kennt
die Attribute ihrer Klasse und kopiert rekursiv alle referenzierten Objekte auf
den neuen Heap. So wird ein Baum aufgespannt, der in der Wurzelmenge beginnt.
Dieser stellt alle erreichbaren Objekte dar.

%TODO: Baumgraph mit Wurzelmenge (Stack) auf Heap zeigen

\section{Vorbereitung}
% 1. Heap teilen
% 2. _ch, _nh
% 3. NEW nullt Attribute
% 4. Lokale Variablen nullen (VarOrCall)
% 5. Wurzelmenge (R4)
% 6. Fixes fuer R4 (_free und changes im code)
% 7. NEW ruft _lookup auf, die die Obergrenzen ueberprueft und den Garbage
%    Collector in Gang setzt, falls notwenig
% 8. Reihenfolge aendern fuer AccessSteatement, damit Objekt-Referenz nicht auf R2 liegt.
% 9. Objekte um _newAddress erweitern
% 10. Klonemethode implementieren
% 11. Code-Fixes, damit clone klappt (box/unbox, isA, CastStatement)
% 12. Attributliste beschaffen (samt der von Parentklasse)



\chapter{Bonus Aufgaben}

\chapter{Anhang}
\section{Anhang}
\begin{Verbatim}
  Ein wenig Text.
  Mehr Text.
  Nice.
\end{Verbatim}

\begin{Verbatim}[frame=single, label=\textit{N}\textbf{ice}!, 
       labelposition=topline, formatcom=\color{orange}, 
       framesep=2mm, framerule=2mm, fontseries=b]
  Etwas Style!
\end{Verbatim}

\textbf{Bold Text}


\end{document}
