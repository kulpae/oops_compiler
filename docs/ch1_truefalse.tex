\section{Einführung}
In diesem Kapitel führen wir die Begriffe \verb|TRUE| und \verb|FALSE| ein.
Da noch keine Boolschen Ausdrücke Exisitieren, entsprechen die dann 1 und 0.\\

\section{Lexikalische Analyse}
Damit die Symbole  \verb|TRUE| und \verb|FALSE|  eingelesen werden können müssen sie zuerst in  \verb|Symbol.java| definiert werden.
\lstinputlisting[firstnumber=21, linerange={21-21}]{../oopsc/src/Symbol.java}
Nun da wir Symbole haben mit denen OOPSC arbeiten kann, definieren wir in \verb|LexicalAnalysis.java|, dass er die Symbole anlegen soll, wenn er \verb|TRUE| bzw. \verb|FALSE| ließt.
\lstinputlisting[firstnumber=89, linerange={89-92}]{../oopsc/src/LexicalAnalysis.java}

\section{Syntaktische Analyse}
In der Syntax von OOPS gehören  \verb|TRUE| und \verb|FALSE| zu den Literalen, also müssen wir in \verb|SyntaxAnalysis.java| Sie als LiteralExpression in \verb|literal()| einfügen. Da nun alle wieteren Komponenten wie BoolType bereits bekannt sind, könne wir sie verwenden und  \verb|TRUE| und \verb|FALSE| sind erfolgreich eingebunden.
\lstinputlisting[firstnumber=646, linerange={646-655}]{../oopsc/src/SyntaxAnalysis.java}
