\section{Einführung}
In diesem Kapitel führen wir die Begriffe \verb|TRUE| und \verb|FALSE| ein.
Da noch keine Boolschen Ausdrücke Exisitieren, entsprechen True und False dann jeweils 1 und 0.\\

\section{Lexikalische Analyse}
Damit die Symbole  \verb|TRUE| und \verb|FALSE|  eingelesen werden können müssen sie zuerst in  \verb|Symbol.java| definiert werden.\\
Dafür wird folgendes der enum Id in \verb|Symbo.java| hinzugefügt:
\lstinputlisting[firstnumber=21, linerange={21-21}]{../oopsc/src/Symbol.java}
In \verb|LexicalAnalysis.java| muss eingefügt werden, dass beim lesen von \verb|TRUE| bzw. \verb|FALSE| die jeweiligen Symbole angelegt werden sollen.\\
Dazu wird LexicalAnalysis in \verb|LexicalAnalysis.java| erweitert:
\lstinputlisting[firstnumber=89, linerange={89-92}]{../oopsc/src/LexicalAnalysis.java}

\section{Syntaktische Analyse}
In der Syntax von OOPS gehören  \verb|TRUE| und \verb|FALSE| zu den Literalen, also werden \verb|TRUE| und \verb|False| in \verb|SyntaxAnalysis.java|  als LiteralExpression in \verb|literal()| einfügen. Da nun alle wieteren Komponenten wie BoolType bereits bekannt sind, sind \verb|TRUE| und \verb|FALSE| funktionsfähig integriert.\\
Dazu wird in literal in \verb|SyntaxAnalysis.java| folgendes hinzugefügt:
\lstinputlisting[firstnumber=646, linerange={646-655}]{../oopsc/src/SyntaxAnalysis.java}
