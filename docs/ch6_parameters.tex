\section{Einführung}
In diesem Kapitel wird Das Program um weitere Klassen erweitert.
Dazu muss die Syntax und der Code von Program erweitert werden.
Auch können nun Boolean und Integer diesen Klassen hinzugefügt werden, statt gesondert behandelt zu werden.

\section{Klassen Deklaration}
Damit alle Deklarationen zur Kontext-Analyse bekannt sind, muss das Deklarieren und die Analyse in ClassDeclaration getrennt werden,
 sodass  eine neue Methode \verb|contextAnalysisForBody(Declaration declarations)| verfügbar ist.
\lstinputlisting[firstnumber=218, linerange={218-231}]{../oopsc/src/ClassDeclaration.java}

\section{mehrere Klassen im Programm}
In Programm muss zuerst die classDeclaration durch eine Liste von classDeclaration's ersetzt werden.
\lstinputlisting[firstnumber=9, linerange={9-12}]{../oopsc/src/Program.java}
Ebenso muss der Konstruktor angepasst werden. In diesem Code werden nun zwei Konstruktoren geführt. Einer für nur eine Klasse und einer führ eine Liste von Klassen.
\lstinputlisting[firstnumber=22, linerange={22-41}]{../oopsc/src/Program.java}
Da nun das Programm bereits mehrere Klassen führen kann, muss nun angepasst werden, was damit getan werden muss.
Dazu ist die Bearbeitung zur vorherigen Klasse ähnlich. Nur dass nun die Deklaration und Kontext-Analyse getrennt ausgeführt werden müssen, damit sämtliche Definitionen der Klassen und Methoden der Klassen wärend der ganzen Kontext-Analyse vorhanden ist.
\lstinputlisting[firstnumber=98, linerange={98-122}]{../oopsc/src/Program.java}

Da Boolean und Integer ebenfalls Klassen sind können sie nun mit den anderen Behandelt werden, so entfällt der bisherige Code in \verb|Program.java| von Integer und Boolean stattdessen werden sie der Liste der Klassen hinzugefügt.
\lstinputlisting[firstnumber=69, linerange={69-70}]{../oopsc/src/Program.java}
Daraufhin müssen sie aber mit \verb|null| initialisiert werden.
\lstinputlisting[firstnumber=86, linerange={86-94}]{../oopsc/src/Program.java}

\section{Syntax}
Zum Schluss muss noch die Syntax angepasst werden, sodass auch mehrere Klassen für ein Programm gelesen werden können.
Dazu fügen wir der Funktion \verb|parse()| eine Schleife hinzu, welche mit Hilfe des Symbols \verb|Class| nach Klassen sucht.
Die Notwendigkeit und Eigenschaften von Main sind hier nicht nötig, da sie wie vorher in \verb|Programm.java| berücksichtigt werden.
\lstinputlisting[firstnumber=742, linerange={742-750}]{../oopsc/src/SyntaxAnalysis.java}

\section{Ausführender Code}
Der meiste Code ist bereits über die Deklarationen abgedeckt, somit sind die Aufrufe bereits abgeschlossen.
Da ebenfalls bereits für Klassen eine vollständige \verb|generateCode| Methode existiert, muss Diese nurnoch für alle Klassen ausgeführt werden.
\lstinputlisting[firstnumber=197, linerange={197-201}]{../oopsc/src/Program.java}
