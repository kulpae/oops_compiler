\section{Einführung}
In diesem Kapitel erweitern wir das If-Statement um einen möglichen else-Pfad und mehreren möglichen elseIf-Pfaden.


\section{Lexikalische Analyse}
Zuerst werde die Symbole in  \verb|Symbol.java| definiert.
\lstinputlisting[firstnumber=22, linerange={22-22}]{../oopsc/src/Symbol.java}
In der lexikalischen Analyse in \verb|LexicalAnalysis.java| wird ergänzt, dass beim einlesen von \verb|ELSE| das Symbol ELSE und bei 
\verb|ELSEIF| das Symbol ELSEIF erzeugt wird.
\lstinputlisting[firstnumber=93, linerange={93-96}]{../oopsc/src/LexicalAnalysis.java}

\section{Syntaktische Analyse}
Bevor mit der syntaktischen Analyse begonnen werden kann muss  ifStatement.java
durch elseStatements erweitert werden. Statements für die ELSEIF-Pfade sind nicht nötig, genauer wird darauf noch später eingegangen.
\lstinputlisting[firstnumber=13, linerange={13-16}]{../oopsc/src/ifStatement.java}
Für spätere Tests ist es auch sinnvoll die print() Funktion zu erweitern.
\lstinputlisting[firstnumber=90, linerange={90-99}]{../oopsc/src/ifStatement.java}
\vspace{3mm}
Die Syntax einer If-Anweisung sieht wie folgt aus:\\ \verb|IF relation THEN statements { ELSEIF relation THEN statements }|\\
\verb|[ ELSE statements ] END IF|\\  bisher haben wir: \verb| IF relation THEN statements END IF|.\\
Als erstes müssen die IF-Statements zus"atzlich bei ELSE, bzw ELSEIF abgebrochen werden.
Entweder man ersetzt dafür die Statements hier durch einen neuen Statement-Typ oder wie in unserem Falle, wird Statements erweitert, da ELSE und ELSEIF in keinen anderen Kontext Verwendung findet.
\lstinputlisting[firstnumber=361, linerange={361-368}]{../oopsc/src/SyntaxAnalysis.java}
Die restlichen Erg"anzungen der SyntaxAnalyse finden in \verb|statement()| unter dem Fall \verb|IF| statt.
Nach dem hinzufügen der IF-Statements muss nun überprüft werden, was als nächstes Zeichen folgt, ist es ELSE, bzw. ELSEIF, muss es verbraucht werden und weitere Statements für die ELSE-Statements, bzw ELSEIF-Statements hinzugefügt werden.
Dabei ist zu beachten, das nur ELSEIF folgende ELSEIF und/oder ELSE zulässt.
Da ELSEIF ebenso wie IF im ELSE-Pfad funktioniert, kann es auch so verwendet werden.
Dazu ver"andert man Das Symbol ELSEIF zu IF, erzeugt daraus ein \verb|ifStatement()| und fügt dieses zu den elseStatements hinzu.
\lstinputlisting[firstnumber=395, linerange={395-399}]{../oopsc/src/SyntaxAnalysis.java}
Dadurch werden Rekursiv weitere ELSEIF-Pfade möglich.
Falls jedoch kein ELSEIF noch vorhanden ist, muss noch der ELSE-Fall geprüft werden.
Wenn das Symbol ELSE erkannt wird, werden die folgenden Statements den ELSE-Statements hinzugefügt.
\lstinputlisting[firstnumber=400, linerange={400-404}]{../oopsc/src/SyntaxAnalysis.java}

\section{Kontext Analyse}
Die Kontext Analyse findet in der Regel in der dazugehörigen Java-Datei statt.
In ifStatement.java ist bereits eine Analyse für das Statement mit IF-Pfad vorhanden. Diesen müssen wir nun noch mit dem ELSE-Pfad ergänzen. Da wir IFELSE ersetzen ist keine eigene Analyse mehr nötig.
\lstinputlisting[firstnumber=39, linerange={39-43}]{../oopsc/src/ifStatement.java}

\section{Ausführender Code}
Auch hier wird auf den bereits bestehenden Code aufgesetzt.
Zuerst brauchen wir ein weiteres Label für den ELSE-Pfad.
\lstinputlisting[firstnumber=109, linerange={109-109}]{../oopsc/src/ifStatement.java}
Der Sprung falls die Bedingung nicht zutrifft muss auf den ELSE-Pfad gehen.
\lstinputlisting[firstnumber=116, linerange={116-116}]{../oopsc/src/ifStatement.java}
Um einen Code-Block zu erzeugen, der im normalen Fluss nicht erfasst wird, wird nach den IF-Statements ein Sprung zum "If END"-Label hinzugefügt. Damit dieser Block nun als ELSE-Behandlung erkannt wird, kommt das ELSE-Label hinzu.
Nun folgen die ELSE-Statements.
\lstinputlisting[firstnumber=121, linerange={121-128}]{../oopsc/src/ifStatement.java}